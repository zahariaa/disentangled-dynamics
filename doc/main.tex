\documentclass[10pt,letterpaper]{article}

\usepackage{ccn}
\usepackage{pslatex}
\usepackage{apacite}

\title{Analyzing disentanglement of visual objects in semi-supervised neural networks}
 
\author{{\large \bf Andrew David Zaharia (andrew.z@columbia.edu)}$^{1,\star}$
  \AND {\large \bf Benjamin Peters (benjamin.peters@columbia.edu)}$^{1,\star}$
  \AND {\large \bf John Cunningham (jpc2181@columbia.edu)}$^2$
  \AND {\large \bf Nikolaus Kriegeskorte (n.kriegeskorte@columbia.edu)}$^{1,3}$
  \AND $^1$ Mortimer B. Zuckerman Mind Brain Behavior Institute\\
  Columbia University, New York, NY 10027, USA\\
  $^2$ Department of Statistics and Grossman Center\\
  Columbia University, New York, NY 10027, USA\\
  $^3$ Departments of Psychology, Neuroscience, and Electrical Engineering\\
  Columbia University, New York, NY 10027, USA
  \AND $^\star$ These authors contributed equally to this work}

%% AZ additions
\usepackage{graphicx,xcolor}
\graphicspath{{./figures/}}

% \hyphenation{auto-encoder}
\hyphenation{auto-encoders}
% \hyphenation{dis-entangle}
\newcommand{\bvae}{$\beta$-VAE~}

\begin{document}

\maketitle

\section{Abstract}
{
\bf
A fundamental goal of visual systems is to condense visual stimuli into compact representations of relevant information they contain. Ideally, these representations would consist of the independent ``generative factors'' that fully determine, on a semantic level, the visual input. Such a ``disentangled'' representation could consist of the identity of a background scene, and the identity, position, pose, and size of an object. Recent research in deep neural networks (DNNs) has focused on achieving disentangled representations, through unsupervised learning, of single objects or faces in isolation. We trained and analyzed a popular DNN model of disentanglement, the $\beta$-variational autoencoder ($\beta$-VAE) on a new dataset, containing a ``foreground'' white circle and ``background'' isotropic Gaussian. We show that the neural network autoencoder architecture we use can achieve a perfectly disentangled latent representation with supervised learning, but only achieves partial disentanglement when using the unsupervised \bvae loss function. On our dataset, higher $\beta$ results in higher reconstruction loss and greater entanglement. We propose that further inductive bias is needed to achieve better disentanglement, such as a representation which factorizes static properties and their dynamics.
}
\begin{quote}
\small
\textbf{Keywords:} 
disentanglement; unsupervised learning; deep neural network; autoencoder; object vision
\end{quote}


\section{Introduction}

 \textit{Disentangled} visual representations are ones which compactly and independently encode the true generative factors of the world \cite{DiCarlo2007,Bengio2009}. In object recognition for example, such factors could be the object's identity, size, rotation, lighting, and color.

 Recent work on disentanglement in DNNs has focused on the unsupervised learning setting. The $\beta$-variational autoencoder (\bvae) claimed to learn more disentangled representations by treating the reconstruction error term in the VAE loss function as a regularizer \cite{Higgins2017,Kingma2014}. This approach, however, is quite general and do not necessarily lead to truly disentangled representations.


\section{Experiments}

% \begin{figure}[h!]
%   % \begin{center}
%      \includegraphics[width=3.375in]{latents_smaller.pdf}
%   % \end{center}
%   \caption{\textbf{Circle+Gaussian dataset and varying levels of entangled representations.}}{(a) The circle+Gaussian dataset, with one generating factor changing in each column (from left to right: circle horizontal and vertical locations, Gaussian horizontal and vertical locations). (b) A perfectly disentangled representation. As one generative factor linearly increases, one unique latent variable also linearly increases while the rest are constant. (c) In entangled representations, as one generative factor changes, multiple latent factors change.}
%   \label{fig:dataset}
% \end{figure}

We assessed the level of disentanglement existing models can achieve in extremely simple settings where perfect disentanglement should be achievable by creating a new, simple dataset. We randomly varied the positions of one circle and one Gaussian while keeping size and intensity fixed. Therefore, there are 4 generative factors for this dataset: the horizontal and vertical position of each object. An ideal encoder for these images that is perfectly disentangled is one with four latent variables, where each one uniquely maps to one of the four generative factors. An entangled representation would be one in which multiple latent variables change when varying a single generative factor, and potentially nonlinearly.

As a basic control, we wanted to verify that a perfectly disentangled encoder could be achieved by training a simple encoder network with supervision to map its four latent variables directly to the generative factors. We found that this architecture is capable of representing the underlying generative factors in a perfectly disentangled way.

Next, we trained \bvae with the same encoder network architecture and a decoder network with the size-matched fully connected and deconvolutional layers in reverse, for different $\beta$ values. The resulting representations are entangled, and become less informative for higher $\beta$ values. The level of disentanglement and reconstruction quality in \bvae further declined with increasing $\beta$. This is somewhat at odds with previous predictions, in which one should trade-off with the other \cite{Higgins2017,Alemi2017}, but is likely due to our choice of a low number of latent variables.

Objects are not static; they exist in a continuous world. The putative generative factors that determine an object’s appearance are likely to remain stable or vary smoothly and slowly over time \cite{Wiskott2002}. This inductive bias could support feature learning in biological systems. We will train and analyze an autoencoder that that a representation which factorizes static properties and their dynamics will lead to better disentanglement.

\bigskip

%\section{Acknowledgments}
%Place acknowledgments (including funding information) in a section at
%the end of the paper.

\bibliographystyle{apacite}
\setlength{\bibleftmargin}{.125in}
\setlength{\bibindent}{-\bibleftmargin}
\bibliography{refs}

\end{document}
