\documentclass{article}

% if you need to pass options to natbib, use, e.g.:
%     \PassOptionsToPackage{numbers, compress}{natbib}
% before loading neurips_2018

% ready for submission
% \usepackage{neurips_2018}

% to compile a preprint version, e.g., for submission to arXiv, add add the
% [preprint] option:
     \usepackage[preprint]{neurips_2018}

% to compile a camera-ready version, add the [final] option, e.g.:
     %\usepackage[final]{neurips_2018}

% to avoid loading the natbib package, add option nonatbib:
%     \usepackage[nonatbib]{neurips_2018}

\usepackage[utf8]{inputenc} % allow utf-8 input
\usepackage[T1]{fontenc}    % use 8-bit T1 fonts
\usepackage{hyperref}       % hyperlinks
\usepackage{url}            % simple URL typesetting
\usepackage{booktabs}       % professional-quality tables
\usepackage{amsfonts}       % blackboard math symbols
\usepackage{nicefrac}       % compact symbols for 1/2, etc.
\usepackage{microtype}      % microtypography

\usepackage{amsmath}      % microtypography

\title{Disentangling via dynamics}

% The \author macro works with any number of authors. There are two commands
% used to separate the names and addresses of multiple authors: \And and \AND.
%
% Using \And between authors leaves it to LaTeX to determine where to break the
% lines. Using \AND forces a line break at that point. So, if LaTeX puts 3 of 4
% authors names on the first line, and the last on the second line, try using
% \AND instead of \And before the third author name.

\author{%
  Benjamin \& Andrew \\ %\thanks{}
  %Zuckerman Institute\\
  %Columbia University\\
  %New York, NY 10027 \\
  % examples of more authors
  % \And
  % Coauthor \\
  % Affiliation \\
  % Address \\
  % \texttt{email} \\
  % \AND
  % Coauthor \\
  % Affiliation \\
  % Address \\
  % \texttt{email} \\
  % \And
  % Coauthor \\
  % Affiliation \\
  % Address \\
  % \texttt{email} \\
  % \And
  % Coauthor \\
  % Affiliation \\
  % Address \\
  % \texttt{email} \\
}

\begin{document}
% \nipsfinalcopy is no longer used

\maketitle

\begin{abstract}


\end{abstract}

\section{Introduction}

A fundamental goal in machine learning is to build algorithms which can learn to predict future states, given a set of training data. In order to make such a problem tractable, certain inductive biases are adopted which constrain the possible solutions to a smaller set. In the case of visual object recognition and classification, one commonly adopted inductive bias is that the appearance of a given object can be determined by a small set of independent generative factors, such as the object’s identity, size, rotation, lighting, and color. A learned representation having these properties is said to be \textit{disentangled} \cite{DiCarlo2007,Bengio2009}, that is, object properties with different semantic meanings are not distributed across latent variables.

Recent work has attempted to develop variational autoencoders (VAE) \cite{Kingma2014} that can learn disentangled representations, either by regularizing reconstruction error \cite{Higgins2017}, targeting a particular point on a rate-distortion curve \cite{Alemi2017}, or level of overlap of latent factors \cite{Mathieu2018}. These approaches, however, are quite general and do not necessarily lead to truly disentangled representations \cite{Alemi2017,Mathieu2018}.

We hypothesize that a further inductive bias is necessary to produce truly disentangled representations. Objects are not static; they exist in a continuous world. The putative generative factors that determine an object’s appearance are likely to remain stable (in the case of object identity) or vary smoothly and slowly over time (in the case of position, rotation, etc.) \cite{Wiskott2002}. We propose that a representation which factorizes static properties and their dynamics will lead to better disentanglement.

\subsection*{TODO: motivate methods with related work}
\begin{itemize}
    \item briefly discuss Archer paper \cite{Archer2016}
    \begin{itemize}
        \item even more briefly discuss Johnson paper \cite{Johnson2016}
        \begin{itemize}
            \item They have LDS SVAE, which uses structured VAE to model a latent LDS graphical model. SLDS VAE uses latent switching LDS to represent ``continuous latent states that evolve according to a discrete library of linear dynamics''
        \end{itemize}
    \end{itemize}
    \item Related work: \cite{Gregor2015a,Krishnan2015,Watter2015,Karl2016,Fraccaro2016,Fraccaro2017a,Krishnan2017}
    \item Krishnan2015 ``Deep Kalman Filters''
    \begin{itemize}
        \item Use MLP or RNNs rather than PGMs
        \item Very similar. ELBO + Kalman filter
        \item Uses NNs for nonlinear dynamics
    \end{itemize}
    \item Krishnan et al 2017 ``Structured Inference Networks for Nonlinear State Space Models'' further generalizes this. Still using RNNs.
    \item Fraccaro et al 2017 ``A Disentangled Recognition and Nonlinear Dynamics Model for Unsupervised Learning''
    \begin{itemize}
        \item Separate latent representations: object rep from recognition network, and a latent state describing its dynamics
        \item ``Kalman VAE''
        \item ``At each time step $t$, a variational auto-encoder compresses high-dimensional visual stimuli $x_t$ into latent encodings $a_t$. The temporal dynamics in the learned $a_t$-manifold are modelled with a linear Gaussian state space model that is adapted to handle complex dynamics (despite the linear relations among its states $z_t$). The parameters of the state space model are adapted at each time step, and non-linearly depend on past $a_t$’s via a recurrent neural network''
        \item They model a bouncing ball ``video''. I think we can do better?
    \end{itemize}
    \item Fraccaro et al 2016 ``Sequential Neural Models with Stochastic Layers''
    \begin{itemize}
        \item Has separate layers of deterministic and stochastic latent variables
    \end{itemize}
    \item Gregor 2015 ``DRAW'' uses RNN to refine image reconstructions sequentially
\end{itemize}

\subsection*{TODO: briefly summarize our contributions and tasks}




\section{Methods}

Let $\mathbf{x} = (\mathbf{x}_1, \dots, \mathbf{x}_t, \dots, \mathbf{x}_T)$ the observations and $\mathbf{z} = (\mathbf{z}_1, \dots, \mathbf{z}_t, \dots, \mathbf{z}_T)$ the corresponding latent embeddings, where $\mathbf{x}_t \in \mathbb{R}^m$ and $\mathbf{z}_t \in \mathbb{R}^n$.

Following \cite{Archer2016} we structure the inverse covariance $\Sigma^{-1}$ of the approximate posterior $q(z|x) = \mathcal{N}(\mu_\phi(x), \Sigma_\phi(x))$ to have a blog tri-diagonal structure.

$$
\Sigma_\phi(x)^{-1} = \begin{bmatrix}
  D_0 & B_0^T & & \\
  B_0 & D_1 & B_1^T & \\
   & \ddots & \ddots & B^T_{T-1} \\
   & & B_{T-1} & D_T \\
\end{bmatrix}
$$

This particular structure embodies that $z_t$ conditionally only depends on $z_{t-1}$ described by the precision matrix $B_0$.

The matrices $B$ correspond to the partial correlations of latent variables between adjacent time-points. We here propose to impose linear dynamics for transitions in latent space

\begin{align*}
  \mathbf{z}_t = A_t \mathbf{z}_{t-1} \\
\end{align*}

where $A_t = L \Lambda_t L^{-1}$ is the dynamics matrix with eigenvector basis $L$ and eigenvalues $\Lambda_t$ at time-point $t$. In particular, for factorized latent representations we set $L = I$. 

Allowing different eigenvalues for each time-point $t$ admits complex dynamics. We will later impose regularizing constraints on the variation of the eigenvalues across time-points.

The approximate posterior for a given time point $t$ is an isotropic Gaussian (following standard variational mean-field approximation), thereby reflecting the conditional independence assumption that $q(z_t|x_t) = \prod_{i}^n q(z^{(i)}_t|x_t)$. We therefore set $D_t$ to I. [\textit{unclear on that part}]

\subsection{Unconstrained Eigenvalues}

In the first case, eigenvalues of $A_t$ can change with every time-point and the mean and eigenvalues are parametrized by a neuronal network:

\begin{align*}
    \Lambda_t & = \text{NN}_{\phi_\Lambda}(\textbf{x}_t, \textbf{x}_{t-1}) \\
    \mu_t & = \text{NN}_{\phi_\mu}(\textbf{x}_t) \\
\end{align*}

\subsection{Constraining dynamics}

\subsubsection{Slow dynamics}

To encourage slow dynamical changes in latent space (and therefore learning of slow changing/predictable features) wen can  constraint the transitional dynamics to be slow. In the first step we therefore add a regularizing term to the loss:

\begin{align*}
    \sum_0^{T-1} ||\text{diag}(\Lambda_t) - \mathbf{1}||_1
\end{align*}

\subsubsection{Different dynamics for latents}

Instead of applying the same regularization to all eigenvalues, we can instead specify a prior over eigenvalues. In particular, let $\lambda^{(i)}_t$ be the eigenvalue corresponding to the $i$-th latent dimension for the transition from time-point $t$ to $t+1$. For instance, defining the regularizing loss as 

\begin{align*}
    \sum_0^{T-1} \sum_i^{n} \gamma(i) |\lambda^{(i)}_t - 1|
\end{align*}

where the regularizing hyperparameter $\gamma$ allows for different strengths of the 'slow' dynamics regularizer for each latent variable (e.g., $\gamma(i) \propto i$). (\textit{this probably imposes an implicit prior on lambda - there is probably a better way to get this prior than just tuning the regularization.}).

\subsection{Future directions or additional parts}
Depending on how well things go (and make sense), this can be the next steps:
\subsubsection{Allowing for interactions in latent space}

Some generative factors can only be expressed in more than one dimension (e.g. angle of rotation of an object). The diagonal construction of A might therefore be to constrained.

Hence, the idea would be to relax $A$ to deviate from diagonal, hence $A_t = \text{NN}(\mathbf{x_t})$ thereby allowing for interactions in latent space, while still regularizing:

\begin{align*}
    \sum_0^{T-1} ||A_t - I||_1
\end{align*}

\subsubsection{Enforcing low-dimensional subspaces}

What we actually want (instead of having an A matrix that has small deviations from an identity matrix) are small subspaces (i.e., small Jordan blocks in the Jordan decomposition of A). Therefore, this step would try to actually enforce this constraint directly.

\subsection{Assessment of disentanglement}

\subsubsection{Qualitative}

\begin{itemize}
\item Are two generative objects (i.e., an object vs. the background) that change with different dynamics captured in different latent dimensions? 
\item Can we recover bimodal distributions of learned eigenvalues?
\end{itemize}

\subsection{Quantitative}

Using established metrics, we will compare the disentanglement of latent representations across the different model variants as well as with established models in the literature.

missing part: construcing the full covariance matrix from


\section{Results}


\section{Discussion}

\bibliography{refs}
\bibliographystyle{unsrt}

\end{document}
