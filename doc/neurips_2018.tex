\documentclass{article}

% if you need to pass options to natbib, use, e.g.:
%     \PassOptionsToPackage{numbers, compress}{natbib}
% before loading neurips_2018

% ready for submission
% \usepackage{neurips_2018}

% to compile a preprint version, e.g., for submission to arXiv, add add the
% [preprint] option:
    \usepackage[preprint]{neurips_2018}

% to compile a camera-ready version, add the [final] option, e.g.:
%     \usepackage[final]{neurips_2018}

% to avoid loading the natbib package, add option nonatbib:
%     \usepackage[nonatbib]{neurips_2018}

\usepackage[utf8]{inputenc} % allow utf-8 input
\usepackage[T1]{fontenc}    % use 8-bit T1 fonts
\usepackage{hyperref}       % hyperlinks
\usepackage{url}            % simple URL typesetting
\usepackage{booktabs}       % professional-quality tables
\usepackage{amsfonts}       % blackboard math symbols
\usepackage{nicefrac}       % compact symbols for 1/2, etc.
\usepackage{microtype}      % microtypography

\title{Using prediction at multiple timescales for object representation learning}

% The \author macro works with any number of authors. There are two commands
% used to separate the names and addresses of multiple authors: \And and \AND.
%
% Using \And between authors leaves it to LaTeX to determine where to break the
% lines. Using \AND forces a line break at that point. So, if LaTeX puts 3 of 4
% authors names on the first line, and the last on the second line, try using
% \AND instead of \And before the third author name.

\author{%
  Andrew D~Zaharia\thanks{Use footnote for providing further information
    about author (webpage, alternative address)---\emph{not} for acknowledging
    funding agencies.} \\
  Zuckerman Mind Brain Behavior Institute\\
  Columbia University\\
  New York, NY 10027 \\
  \texttt{andrew.z@columbia.edu} \\
}

%%%%%%%%%%%%%%%%%%%%%%%%%%%%%%%%%%%%%%%%%%%%%%%%%%%%%%%%%%%%%%%%%%%%%%%%%%%%%%%%
\begin{document}
% \nipsfinalcopy is no longer used

\maketitle

\begin{abstract}
  Coming soon.
\end{abstract}


\section{Introduction}

How to best learn representations of the environment that are useful for solving tasks is a fundamental question in neural network research. State-of-the-art techniques for many machine learning tasks, such as object classification, involve supervised learning. Machine performance on these tasks lags behind human performance---they generally don't learn representations of object categories in the same, fully supervised manner. Yet they do not learn in a fully unsupervised manner either: in addition to rewards in the environment, there is a wealth of information that is contained in the stimuli themselves.

\section*{Outline of the rest of the introduction}
\subsection*{the rest of the introduction in the introduction}
\begin{itemize}
  \item Identities of objects and scenes in the real world tend to be relatively stable, while their appearance in pixel space can change drastically (e.g., lighting, rotation, scaling, within-category variation). cf. Slow feature analysis \cite{Wiskott2002}
  \item Want disentangled representations \cite{DiCarlo2007,Bengio2009,Chen2016,Higgins2017,Alemi2017}
  \item Making predictions, and comparing those predictions to incoming stimuli can serve as a self-supervision signal \cite{Whitney2016,Doersch2015}
\end{itemize}
\subsection*{Recent work}
\begin{itemize}
  \item Contrastive Predictive Coding (CPC) \cite{Oord2018} uses prediction as a self-supervision signal to attempt to learn more "useful" object representations.
  \item Furthermore, CPC uses an objective function that implicitly maximizes the mutual information between latent "context" variables and \emph{future} inputs, by explicitly minimizing normalized density ratios of inferred \emph{future latent} variables and samples from the posterior.
  \item CPC makes predictions at one level of abstraction, and on immediately subsequent instances in input stream
\end{itemize}
\subsection*{Our Contribution}
\begin{itemize}
  \item Making extremely precise predictions of the future (e.g., pixel-level image predictions) only makes sense for very short time intervals, and becomes impossible for longer time scales
  \item we aim to make predictions on multiple time scales, with increasing levels of abstraction and complexity for increasing time horizons
  \item As yet not determined method: we will implement a hierarchical, autoregressive model (e.g. ResNet VAE \cite{Kingma2016}, Ladder VAE \cite{Sonderby2016}, Variational Ladder Autoencoder \cite{Zhao2017}, etc) with a CPC-like objective that makes predictions on latent variables at multiple levels of abstraction
\end{itemize}
\subsection*{Possible tasks}
\begin{itemize}
  \item A sequence of frames in which a subset MNIST digits appear at random locations from frame to frame, but the same subset of digits have a very high probability of occurring in the subsequent frame.
  \begin{itemize}
    \item e.g., 1 and 2 each have a 99\% probability of being in a given frame, and 3 and 4 have a 1\% probability of occurring in a given frame
    \item the point of this task is to dissociate location cues with object identity
    \item we want the model to only represent and synthesize the objects that predictably appear, and ignore the ``distractor'' objects
  \end{itemize}
  \item self-supervised training using the CORe50 dataset \cite{Lomonaco2017}, which has 15sec 20fps videos of 50 objects, held in hand, belonging to 10 categories, each recorded in 11 distinct sessions (8 indoor, 3 outdoor). test object classification of new instances in the dataset
  \item A similar task on Basel faces \cite{Paysan2009}, as in \cite{Kulkarni2015,Whitney2016}
\end{itemize}

\bibliography{refs}
\bibliographystyle{unsrt}

\end{document}
